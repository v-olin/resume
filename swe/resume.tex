\documentclass[11pt,oneside,a4paper,titlepage]{article}

\usepackage[most]{tcolorbox}
\usepackage{geometry}
\usepackage{tgbonum}
\usepackage{graphicx}
\usepackage{tikz}
\usepackage{hyperref}

\geometry{
    a4paper,
    left    =-0.1cm,
    right   =-0.1cm,
    top     =-0.1cm,
    bottom  =-0.1cm
}

\newsavebox{\picbox}

\newcommand{\cutpic}[3]{
  \savebox{\picbox}{\includegraphics[width=#2]{#3}}
  \tikz\node [draw, rounded corners=#1, line width=1pt,
                color=white, minimum width=\wd\picbox,
                minimum height=\ht\picbox, path picture={
                \node at (path picture bounding box.center) {
                    \usebox{\picbox}};
                }
            ]{};
}

\definecolor{titleBack}{RGB}{47,133,194}
\definecolor{sup-lgray}{RGB}{238,238,238}

\title{Victor Olin}
\date{}

\begin{document}

\tcbset{colframe=white,colback=titleBack,arc=0mm}

\begin{tcolorbox}
    \begin{minipage}{4.5cm}
        \cutpic{2cm}{4cm}{pfp.jpg}
    \end{minipage}
    \begin{minipage}{15cm}
        \begin{center}
            {\fontfamily{qag}\selectfont
                \Huge{\textcolor{white}{Victor Olin}} \\
                \vspace*{0.5cm}
                \Large{\textcolor{white}{Datateknik student \textbar{ } Programmerings entusiast}}
            }
        \end{center}
    \end{minipage}
\end{tcolorbox}

% \titleformat{\section*}[display] {\fontfamily{qag}\selectfont}

\tcbset{colframe=white,colback=white,arc=0mm,fit to height=24cm}
\begin{tcolorbox}
    \begin{minipage}[t]{8cm}
        \vspace*{-18.2pt}
        \begin{tcolorbox}[grow to left by=15pt,colframe=sup-lgray,colback=sup-lgray,sharp corners]
            {\fontfamily{qag}\selectfont
                \scalebox{1.5}{\textbf{Bakgrund}}\\\\
                Jag är en positiv, ambitiös och driven\\
                person med ett stort intresse för\\
                teknik. Alltifrån serverhallar till\\
                flygplan. Under min gymnasiet fick jag\\
                ett stort intresse för datorer och mer\\
                specifikt programmering.
                Efter\\gymnasiet tog jag ett sabbatsår\\
                och jobbade på utbildningskoncernen\\
                AcadeMedia, där jag fick utveckla\\
                flera program inom diverse områden
                för att underlätta arbetsflöden. Mina\\
                tidigare chefer brukar beskriva
                mig som ansvarstagande och har berömt\\
                mig för min förmåga att passa in i\\
                olika miljöer och skapa god stämning\\
                på jobbet.\\\\
                \scalebox{1.5}{\textbf{Kontakt}}\\
                \vspace*{15pt}
                \begin{tabular}{l l}
                    \\Telefon: & 070-975 53 52 \\
                    E-mail: & victor.vieweg@icloud.com \\
                    Github: & \href{https://github.com/v-olin}{https://github.com/v-olin} \\
                    Hemsida: & \href{https://www.v-olin.dev}{https://www.v-olin.dev} \\
                    % \vspace*{10pt}
                \end{tabular} \\
                \scalebox{1.5}{\textbf{Kunskaper}}
                \begin{itemize}
                    \item{God kunskap inom .NET och\\programmering i C\#}
                    \item{Erfarenhet inom andra språk\\som C, Java, JavaScript,\\Haskell och Python}
                    \item{Väldigt stresstålig och bra\\lagkamrat}
                    \item{Genuint intresserad av att\\utvecklas och lära mig om\\nya områden}
                    \vspace*{10pt}
                \end{itemize}
                \scalebox{1.5}{\textbf{Intressen}}
                \begin{itemize}
                    \item{Flygplan och flygsimulatorer}
                    \item{Allt gällande datorer \& \\programmering}
                    \item{Golf\\\\\\\\\\}
                \end{itemize}
            }            
        \end{tcolorbox}
    \end{minipage}
    \begin{minipage}[t]{12cm}
        \vspace*{-18.2pt}
        \begin{tcolorbox}[grow to left by=-10pt,colframe=white,colback=white,sharp corners]
            {\fontfamily{qag}\selectfont
                \scalebox{1.5}{\textbf{Utbildning}}\\
                \\\textbf{Civilingenjörsexamen i Datateknik, Chalmers} \\
                \emph{2020 - 2025} \\
                I hösten av 2020 påbörjade jag min utbildning\\
                för att bli civilingenjör med inriktning på\\
                datateknik. Jag har alltid varit fascinerad av\\
                datorer enda sedan jag var liten och under\\
                mina första två år har jag lärt mig otroligt\\
                mycket och upptäckt nya områden som jag har\\
                blivit passionerad om.\\\\
                \textbf{Teknikprogrammet på NTI Gymnasiet, Stockholm} \\
                \emph{2016 - 2019} \\
                I gymnasiet studerade jag det högskoleinriktade\\
                teknikprogrammet med sikte på att bli civilingenjör\\
                inom datateknik. Under mina tre år lästa jag\\
                utökade studier inom engelska, 3D-modellering\\
                och avancerad matematik.\\\\\\
                \scalebox{1.5}{\textbf{Arbetserfarenhet}}\\
                \\\textbf{IT tekniker, AcadeMedia Support}\\
                \emph{2019 - 2020}\\
                Efter gymnasiet tog jag ett sabbatsår och jobbade\\
                som teknisk support på nordens största\\
                friskolekoncern AcadeMedia. Det var en mycket\\
                lärande möjlighet och jag fick ta del av hur\\
                arbetslivet ser ut inom IT och vilka utmaningar\\
                som man ställs inför dagligen inom IT och\\
                infrastruktur. Jag fick även bidra till större\\
                projekt under min tid gällande programmering\\
                av skräddarsydda  verktyg och felsökning av\\
                program hos nationella IT-leverantörer.\\\\
                \\\textbf{Fastighetstekniker, Fabege AB}\\
                \emph{2019, 2021}\\
                Under sommrarna 2019 och 2021 arbetade jag som\\
                fastighetstekniker på Fabege i Arenastaden. Jag\\
                fick lära mig mycket om hur driftsystem ser ut\\
                i praktiken och vilka eventuella begränsningar\\
                och riktmärken som finns. Det var även mycket\\
                problemlösning inom olika områden från interna\\
                servrar till fläktrum.
            }
        \end{tcolorbox}
    \end{minipage}
\end{tcolorbox}

\newpage

\tcbset{colframe=white,colback=titleBack,arc=0mm}
\begin{tcolorbox}[height from=0cm to 10cm]
    \begin{minipage}[t]{4.5cm}
        \cutpic{2cm}{4cm}{pfp.jpg}
    \end{minipage}
\end{tcolorbox}

\tcbset{colframe=white,colback=white,arc=0mm}
\begin{tcolorbox}
    \begin{minipage}[t]{2cm}
        \begin{tcolorbox}[colframe=white,colback=white]
            
        \end{tcolorbox}
    \end{minipage}
    \begin{minipage}[t]{15cm}
        \begin{tcolorbox}[colframe=white,colback=white]
            {\fontfamily{qag}\selectfont
                \scalebox{1.5}{\textbf{Övriga meriter}}\\
                \\\textbf{Aktiv medlem i Spånga Scoutkår i 10 år}\\
                Jag har lärt mig mycket under mina år i scouterna,\\
                bland annat ledarskap, överlevnad i naturen och\\
                lagarbete under tuffa förhållanden. Jag har deltagit\\
                i nattliga seglingar til och från Gotland under\\
                större stormar, skidat 20 mil i fjällen under\\
                stormvarning samt varit patrulledare och därmed\\
                hjälpt yngre scouter till rätta. Jag har även\\
                haft ansvar med att jobba i pass under långa\\
                seglingar och hajker.\\\\
                Tävlat och vunnit flertal tävlingar inom\\
                luftgevärsskytte. Jag vann Stockholms\\
                distriktsmästerskap 2015 och kom 9nde plats i\\
                SM samma år.\\\\
                Jag innehar B-körkort för både manuell och automat.\\

                Jag talar flytande svenska och engelska.\\

                Referenser finns att lämnas på begäran.\\

                Jag har skrivit detta CV i \LaTeX och källkod kan\\
                \underline{\href{https://github.com/v-olin/resume}{hittas här}}!
            }
        \end{tcolorbox}
    \end{minipage}
\end{tcolorbox}

\end{document}

        